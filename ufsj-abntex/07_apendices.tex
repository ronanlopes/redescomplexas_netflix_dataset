\begin{apendicesenv}
	
	% Imprime uma página indicando o início dos apêndices
	\partapendices
	
% ----------------------------------------------------------
\chapter{Links} \label{links}
% ----------------------------------------------------------

\begin{itemize}

	\item \textbf{Aplicação}: a aplicação ficará hospedada temporariamente para avaliação no servidor de endereço público \href{http://3.89.98.144/}{http://3.89.98.144/}. Para autenticação, foi criado um usuário padrão "admin@admin.com", senha 111111.
	\item \textbf{Vídeo explicativo}: O vídeo de aproximadamente 5 minutos com uma breve explicação está disponível também temporária e publicamente acessíveis através do \textit{link}: \href{https://ronanlopes.me/video-puc.mp4}{https://ronanlopes.me/video-puc.mp4}
	\item \textbf{Apresentação em Slides}: Os slides referentes à apresentação no formato aproximado de 15 minutos estão disponíveis em \href{https://www.ronanlopes.me/apresentacao.pdf}{https://www.ronanlopes.me/apresentacao.pdf}
	\item \textbf{Código-fonte}: Durante o desenvolvimento da aplicação, utilizou-se um repositório no github: \href{https://github.com/ronanlopes/corona\_viewer}{https://github.com/ronanlopes/corona\_viewer}. O código completo da aplicação juntamente com scripts adicionais para coleta e obtenção de resultados podem ser encontrados no \textit{link} \href{https://www.ronanlopes.me/puc-source-code.zip}{https://www.ronanlopes.me/puc-source-code.zip}


\end{itemize}



\end{apendicesenv}