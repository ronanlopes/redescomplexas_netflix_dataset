% ----------------------------------------------------------
% Introdução
% ----------------------------------------------------------
\chapter[Introdução]{Introdução}

Uma das tarefas mais comuns na implementação de algoritmos, com um grande número de aplicações possíveis, a ordenação de elementos em um conjunto se torna também objeto constante de estudo em termos de otimização. As muitas soluções disponíveis, apesar de produzirem o resultado esperado, podem se mostrar indesejadas em termos de desempenho, dependendo do cenário em que são utilizadas.

A fim de se demonstrar alguns desses cenários, bem como poder demonstrar como é feita a implementação, utilização e análise de algoritmos dessa classe, este trabalho tem como proposta um estudo de caso envolvendo os algoritmos de ordenação conhecidos como InsertionSort e MergeSort, bem como o algoritmo híbrido que utiliza ambos, denominado TimSort. O algoritmo que leva o nome de seu idealizador (Tim Peters) foi criado em 2002 para ser usado na linguagem de programação Python, e tem sido o algoritmo de ordenação padrão de Python desde a versão 2.3. Ele atualmente é usado também para ordenar arrays em Java SE 7.

Nas seções seguintes, detalha-se o problema proposto, bem como detalhes dos algoritmos mencionados para solução do problema de ordenação. Posteriormente, para análise, as rotinas são comparadas em um cenário teórico em termos de complexidade e, para conferência em uma aplicação real, os tempos de execução são medidos e analisados para diferentes formatos de entrada.

