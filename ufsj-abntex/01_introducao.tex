% ----------------------------------------------------------
% Introdução
% ----------------------------------------------------------
\chapter[Introdução]{Introdução}

No contexto da ciência da computação, a área de estudo de redes complexas fornece uma abstração para estudo de sistemas onde os elementos desse sistema realizam algum tipo de interação entre si. Essa abstração toma como base o modelo e estrutura de dados dos grafos e possibilita, através da análise de propriedades estruturais e métricas que fornecem indicadores do comportamento da rede, descrever e compreender melhor o sistema como um todo. Como uma área de aplicação multidisciplinar, pode ser utilizada em aplicações biológicas, análise de tráfego de veículos, interações em redes sociais, dentre outros vários exemplos possíveis.

A fim demonstrar um possível caso de uso, o estudo a ser aqui desenvolvido toma como objeto de estudo a base "Netflix Movies and TV Shows"\footnote{\href{https://www.kaggle.com/shivamb/netflix-shows}{https://www.kaggle.com/shivamb/netflix-shows}}, que contém informações relevantes sobre filmes e séries disponíveis no serviço de \textit{streaming} e foi disponibilizada na rede de descoberta de conhecimento Kaggle. Dentre as sugestões de estudo e análise sugeridas na publicação da base, o tópico "Análise de rede de atores/diretores para encontrar insights interessantes" é um exemplo direto de aplicação de estudo da disciplina de redes complexas.

Nas seções seguintes, são detalhadas as propostas de análise e hipóteses a serem testadas na rede mencionada. O estudo desenvolvido tem como ponto de partida uma compreensão rasa acerca da rede tal como descrito no parágrafo anterior e, progressivamente, clarifica melhor o processo de geração, comportamento e predições relativos ao sistema em análise.