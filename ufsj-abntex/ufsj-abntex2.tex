%% abtex2-modelo-trabalho-academico.tex, v-1.7.1 laurocesar
%% Copyright 2012-2013 by abnTeX2 group at http://abntex2.googlecode.com/ 
%%
%% This work may be distributed and/or modified under the
%% conditions of the LaTeX Project Public License, either version 1.3
%% of this license or (at your option) any later version.
%% The latest version of this license is in
%%   http://www.latex-project.org/lppl.txt
%% and version 1.3 or later is part of all distributions of LaTeX
%% version 2005/12\usepackage{array}/01 or later.
%%
%% This work has the LPPL maintenance status `maintained'.
%% 
%% The Current Maintainer of this work is the abnTeX2 team, led
%% by Lauro César Araujo. Further information are available on 
%% http://abntex2.googlecode.com/
%%
%% This work consists of the files abntex2-modelo-trabalho-academico.tex,
%% abntex2-modelo-include-comandos and abntex2-modelo-references.bib
%%

% ------------------------------------------------------------------------
% ------------------------------------------------------------------------
% abnTeX2: Modelo de Trabalho Academico (tese de doutorado, dissertacao de
% mestrado e trabalhos monograficos em geral) em conformidade com 
% ABNT NBR 14724:2011: Informacao e documentacao - Trabalhos academicos -
% Apresentacao
% ------------------------------------------------------------------------
% ------------------------------------------------------------------------

\documentclass[
	% -- opções da classe memoir --
	12pt,				% tamanho da fonte
	openright,			% capítulos começam em pág ímpar (insere página vazia caso preciso)
	oneside,			% para impressão em verso e anverso. Oposto a oneside
	a4paper,			% tamanho do papel. 
	% -- opções da classe abntex2 --
	%chapter=TITLE,		% títulos de capítulos convertidos em letras maiúsculas
	%section=TITLE,		% títulos de seções convertidos em letras maiúsculas
	%subsection=TITLE,	% títulos de subseções convertidos em letras maiúsculas
	%subsubsection=TITLE,% títulos de subsubseções convertidos em letras maiúsculas
	% -- opções do pacote babel --
	english,			% idioma adicional para hifenização
	french,				% idioma adicional para hifenização
	spanish,			% idioma adicional para hifenização
	brazil,				% o último idioma é o principal do documento
	]{ufsj-abntex2}


% ---
% PACOTES
% ---

% ---
% Pacotes fundamentais 
% ---
\usepackage{cmap}				% Mapear caracteres especiais no PDF
\usepackage{lmodern}			% Usa a fonte Latin Modern			
\usepackage[T1]{fontenc}		% Selecao de codigos de fonte.
\usepackage[utf8]{inputenc}		% Codificacao do documento (conversão automática dos acentos)
\usepackage{lastpage}			% Usado pela Ficha catalográfica
\usepackage{indentfirst}		% Indenta o primeiro parágrafo de cada seção.
\usepackage{color}				% Controle das cores
\usepackage[pdftex]{graphicx}
\usepackage{array}
\usepackage{verbatim}
\usepackage{multirow}
\usepackage{footnote}
\usepackage{subfigure}
\usepackage{nameref}
\usepackage{footmisc}
\usepackage{float}
\usepackage{longtable}
\usepackage{caption}

\setlength\extrarowheight{2pt}
% ---
		
% ---
% Pacotes adicionais, usados apenas no âmbito do Modelo Canônico do abnteX2
% Provê somente o comando \lipsum, para gerar texto aleatório.
% Já que estes comandos serão retirados, pode ser excluída também.
% ---
\usepackage{lipsum}				% para geração de dummy text
% ---
% ---
% Pacotes de citações
% ---
\usepackage[brazilian,hyperpageref]{backref}	 % Paginas com as citações na bibl
\usepackage[alf]{abntex2cite}	% Citações padrão ABNT

% --- 
% CONFIGURAÇÕES DE PACOTES
% --- 

% ---
% Configurações do pacote backref
% Usado sem a opção hyperpageref de backref
\renewcommand{\backrefpagesname}{Citado na(s) página(s):~}
% Texto padrão antes do número das páginas
\renewcommand{\backref}{}
% Define os textos da citação
\renewcommand*{\backrefalt}[4]{
	\ifcase #1 %

	\or
		Citado na página #2.%
	\else
		Citado #1 vezes nas páginas #2.%
	\fi}%
% ---


% ---
% Informações de dados para CAPA e FOLHA DE ROSTO
% ---
%\titulo{Heurísticas aplicadas ao problema de projeto de redes de telecomunicações com qualidade de serviço}
\titulo{Netflix Movies \& TV Shows: Estudo, análise e descoberta de conhecimento na rede}
\autor{Antônio Marcos Machado Bernardes\\Ronan José Lopes}
\local{Universidade Federal de São João del-Rei}
\data{2021}
% \orientador{Elisa de Tuler Albergaria}
% \coorientador{Co-orientador} % se não existir, comente essa linha



% ---
% Configurações de aparência do PDF final

% alterando o aspecto da cor azul
%\definecolor{blue}{RGB}{41,5,195}
\definecolor{blue}{RGB}{0,0,0}

% informações do PDF
\makeatletter
\hypersetup{
     	%pagebackref=true,
		pdftitle={\@title}, 
		pdfauthor={\@author},
    	pdfsubject={\imprimirpreambulo},
	    pdfcreator={LaTeX with abnTeX2},
		pdfkeywords={abnt}{latex}{abntex}{abntex2}{trabalho acadêmico}, 
		colorlinks=true,       		% false: boxed links; true: colored links
    	linkcolor=blue,          	% color of internal links
    	citecolor=blue,        		% color of links to bibliography
    	filecolor=magenta,      		% color of file links
		urlcolor=blue,
		bookmarksdepth=4
}
\makeatother

% --- 

% --- 
% Espaçamentos entre linhas e parágrafos 
% --- 

% O tamanho do parágrafo é dado por:
\setlength{\parindent}{1.3cm}

% Controle do espaçamento entre um parágrafo e outro:
\setlength{\parskip}{0.2cm}  % tente também \onelineskip

% ---
% compila o indice
% ---
\makeindex
% ---

\pagestyle{headings}
\setcounter{page}{1}
\pagenumbering{arabic}

% ----
% Início do documento
% ----
\begin{document}

% Retira espaço extra obsoleto entre as frases.
\frenchspacing 

% ----------------------------------------------------------
% ELEMENTOS PRÉ-TEXTUAIS
% ----------------------------------------------------------
 \pretextual

% ---
% Capa
% ---
\imprimircapa
% ---


% ---
% Inserir a ficha bibliografica
% ---
%\include{00a_fichacatalografica}
% ---

% ---
% Inserir errata - Elemento opcional. Caso não utilize, basta comentar a linha
% ---
%\include{00b_errata}
% ---

% ---
% Inserir folha de aprovação
% ---
%% Isto é um exemplo de Folha de aprovação, elemento obrigatório da NBR
% 14724/2011 (seção 4.2.1.3). Você pode utilizar este modelo até a aprovação
% do trabalho. Após isso, substitua todo o conteúdo deste arquivo por uma
% imagem da página assinada pela banca com o comando abaixo:
%
% \includepdf{folhadeaprovacao_final.pdf}
%
\begin{folhadeaprovacao}

  \begin{center}
    {\ABNTEXchapterfont\large\imprimirautor}

    \vspace*{\fill}\vspace*{\fill}
    {\ABNTEXchapterfont\bfseries\Large\imprimirtitulo}
    \vspace*{\fill}
    
    \hspace{.45\textwidth}
    \begin{minipage}{.5\textwidth}
        \imprimirpreambulo
    \end{minipage}%
    \vspace*{\fill}
   \end{center}
    
   Trabalho aprovado. \imprimirlocal, 17 de janeiro de 2014:

   \assinatura{\textbf{\imprimirorientador} \\ Orientador} 
   \assinatura{\textbf{Carolina Ribeiro Xavier} \\ Convidado 1}
   \assinatura{\textbf{Daniel Ludovico Guidoni} \\ Convidado 2}
   %\assinatura{\textbf{Professor} \\ Convidado 3}
   %\assinatura{\textbf{Professor} \\ Convidado 4}
      
   \begin{center}
    \vspace*{0.5cm}
    {\large\imprimirlocal}
    \par
    {\large\imprimirdata}
    \vspace*{1cm}
  \end{center}
  
\end{folhadeaprovacao}

% ---

% ---
% Dedicatória, Agradecimento e Epígrafe
% ---
%\include{00d_pretextual}
% ---


% ---
% inserir o sumario
% ---
\pdfbookmark[0]{\contentsname}{toc}
\tableofcontents*
% ---



% ----------------------------------------------------------
% ELEMENTOS TEXTUAIS
% ----------------------------------------------------------
%\textual

% ----------------------------------------------------------
% Introdução
% ----------------------------------------------------------
\chapter[Introdução]{Introdução}

No contexto da ciência da computação, a área de estudo de redes complexas fornece uma abstração para estudo de sistemas onde os elementos desse sistema realizam algum tipo de interação entre si. Essa abstração toma como base o modelo e estrutura de dados dos grafos e possibilita, através da análise de propriedades estruturais e métricas que fornecem indicadores do comportamento da rede, descrever e compreender melhor o sistema como um todo. Como uma área de aplicação multidisciplinar, pode ser utilizada em aplicações biológicas, análise de tráfego de veículos, interações em redes sociais, dentre outros vários exemplos possíveis.

A fim demonstrar um possível caso de uso, o estudo a ser aqui desenvolvido toma como objeto de estudo a base "Netflix Movies and TV Shows"\footnote{\href{https://www.kaggle.com/shivamb/netflix-shows}{https://www.kaggle.com/shivamb/netflix-shows}}, que contém informações relevantes sobre filmes e séries disponíveis no serviço de \textit{streaming} e foi disponibilizada na rede de descoberta de conhecimento Kaggle. Dentre as sugestões de estudo e análise sugeridas na publicação da base, o tópico "Análise de rede de atores/diretores para encontrar insights interessantes" é um exemplo direto de aplicação de estudo da disciplina de redes complexas.

Nas seções seguintes, são detalhadas as propostas de análise e hipóteses a serem testadas na rede mencionada. O estudo desenvolvido tem como ponto de partida uma compreensão rasa acerca da rede tal como descrito no parágrafo anterior e, progressivamente, clarifica melhor o processo de geração, comportamento e predições relativos ao sistema em análise.

% ----------------------------------------------------------
% Parte de revisãod e literatura
% ----------------------------------------------------------
%\part{Revisão de Literatura}

\chapter{Detalhamento do Problema Proposto}
A proposta do trabalho aqui desenvolvido consiste em implementar os algoritmos em foco (InsertionSort, MergeSorte e, a partir da utilização das funções implementadas nestes, o algoritmo híbrido TimSort). Para testar o funcionamento das rotinas implementadas, algumas entradas a serem processadas devem ser geradas. A fim de explorar diferentes conjunturas, de forma que um algoritmo em particular não se beneficie do formato do conjunto a ser ordenado, os seguintes cenários são determinados para geração das entradas:

\begin{itemize}
	\item Vetor gerado com elementos em posição completamente aleatória
	\item Vetor gerado com elementos em ordem crescente
	\item Vetor gerado com elementos em ordem decrescente
\end{itemize}

Para avaliar também como os algoritmos se comportam à medida que o tamanho da entrada cresce, os vetores gerados possuem tamanhos 32, 64, 1.024, 10.000, 100.000 e 1.000.000. Esse desempenho será medido em termos de tempo de processamento efetivo e também em número de operações relevantes dentro do algoritmo (como trocas de posição e comparação de maior/menor). Como a medida do tempo em um sistema operacional oscila devido à concorrência dos processos pela CPU, para cada um desses contextos serão feitas 10 execuções, com o intuito de se obter uma média mais confiável. Isso gera um total de 180 execuções (3 configurações de entrada x 6 tamanhos de entrada x 10 execuções) para cada algoritmo. 

A partir dos resultados obtidos, será verificado o comportamento assintótico na prática condiz com a análise de complexidade teórica destes algoritmos. Esta tem por finalidade descrever o comportamento de determinada rotina à medida que o tamanho da entrada a ser processada cresce. Além disso, fornece uma análise dos casos de pior, melhor e médio desempenho de cada algoritmo, de forma a auxiliar a escolha de acordo com a característica dos conjuntos de dados que serão efetivamente ordenados na aplicação.


% Aqui entram os capitulos contendo a revisão de literatura
% ---
% Capitulo de revisão de literatura
% ---

\chapter{O Funcionamento dos Algoritmos}

\section{Insertion Sort}

O Insertion Sort é um algoritmo de classificação baseado em comparação local. Nele, um subconjunto do vetor é mantido como ordenado, o qual vai crescendo à medida que o vetor é percorrido e cada elemento é devidamente posicionado. Nesse posicionamento o novo item em foco tem que encontrar seu lugar apropriado e então ser inserido. Dessa característica origina o seu nome Insertion Sort, ou ordenação por inserção.

Em um passo-a-passo, o algoritmo segue as etapas abaixo ao percorrer o vetor:

\begin{itemize}

\item Se for o primeiro elemento, ele já está classificado.
\item Escolha o próximo elemento
\item Compare com todos os elementos na sub-lista classificada
\item Desloque todos os elementos na sub-lista classificada que são maiores que o valor a ser ordenado
\item Insira o valor
\item Repita até que a sub-lista classificada seja toda a lista

\end{itemize}

Um exemplo para ilustrar:

\begin{figure}[!htb]
\centering
\includegraphics[width=9cm]{img/insertion_sort_1.jpg}
\caption{Parte-se de um vetor não ordenado}
\label{fig:insertion1}
\end{figure}

\begin{figure}[!htb]
\centering
\includegraphics[width=9cm]{img/insertion_sort_2.jpg}
\caption{O Insertion Sort compara os dois primeiros elementos}
\label{fig:insertion2}
\end{figure}

\begin{figure}[!htb]
\centering
\includegraphics[width=9cm]{img/insertion_sort_3.jpg}
\caption{Os números 14 e 33 já estão ordenados de forma crescente. Neste momento, o elemento 14 é a sub-lista ordenada.}
\label{fig:insertion3}
\end{figure}

\begin{figure}[!htb]
\centering
\includegraphics[width=9cm]{img/insertion_sort_4.jpg}
\caption{33 e 27 são comparados}
\label{fig:insertion4}
\end{figure}

\begin{figure}[!htb]
\centering
\includegraphics[width=9cm]{img/insertion_sort_5.jpg}
\caption{Neste caso, 33 está na posição incorreta}
\label{fig:insertion5}
\end{figure}

\begin{figure}[!htb]
\centering
\includegraphics[width=9cm]{img/insertion_sort_6.jpg}
\caption{33 e 27 trocam de lugar. Ele também verifica todos os elementos da sub-lista ordenada. Aqui, vemos que a sub-lista classificada tem apenas um elemento 14, e 27 é maior que 14. Portanto, a sub-lista classificada permanece classificada após a troca. }
\label{fig:insertion6}
\end{figure}

\begin{figure}[!htb]
\centering
\includegraphics[width=9cm]{img/insertion_sort_7.jpg}
\caption{Agora 14 e 27 são a sub-lista ordenada. 33 e 10 são comparados}
\label{fig:insertion7}
\end{figure}

\begin{figure}[!htb]
\centering
\includegraphics[width=9cm]{img/insertion_sort_8.jpg}
\caption{Eles não estão na ordem crescente correta}
\label{fig:insertion8}
\end{figure}

\begin{figure}[!htb]
\centering
\includegraphics[width=9cm]{img/insertion_sort_9.jpg}
\caption{Trocam de lugar}
\label{fig:insertion9}
\end{figure}

\begin{figure}[!htb]
\centering
\includegraphics[width=9cm]{img/insertion_sort_10.jpg}
\caption{Entretanto, a sub-lista ainda não está ordenada corretamente, pois 10 ainda está após o 27}
\label{fig:insertion10}
\end{figure}

\begin{figure}[!htb]
\centering
\includegraphics[width=9cm]{img/insertion_sort_11.jpg}
\caption{Então segue a troca}
\label{fig:insertion11}
\end{figure}

\begin{figure}[!htb]
\centering
\includegraphics[width=9cm]{img/insertion_sort_12.jpg}
\caption{14 e 10 estão desordenados, então trocam de logar}
\label{fig:insertion12}
\end{figure}

\begin{figure}[!htb]
\centering
\includegraphics[width=9cm]{img/insertion_sort_13.jpg}
\caption{Ao fim da terceira iteração, temos uma sub-lista ordenada de 4 itens. O processo segue iterativamente.}
\label{fig:insertion13}
\end{figure}

\section{Merge Sort}

O Merge Sort é um dos algoritmos de classificação mais eficientes. Ele tem como princípio de funcionamento a divisão e conquista. O algoritmo divide repetidamente uma lista em várias sublistas até que cada sublista consista em um único elemento, e ao retornar mesclando essas sublistas ordenadamente, obtém-se a lista original de forma ordenada. Esse processo pode ser descrito pelos passos a seguir:

\begin{itemize}

	\item Se houver apenas 1 elemento na lista, está ordenado, retorne
	\item Divida a lista recursivamente ao meio até que não possa ser mais dividida
	\item Mescle as listas menores em uma nova lista ordenada

\end{itemize}

Para ilustrar o funcionamento, toma-se como exemplo de partida o vetor abaixo.

\begin{figure}[!htb]
\centering
\includegraphics[width=9cm]{img/merge1.jpg}
\label{fig:merge1}
\end{figure}

Sabemos que o Merge Sort primeiro divide todo o vetor iterativamente em metades iguais, até que os elementos singulares sejam alcançados. Vemos aqui que um vetor de 8 itens é dividida em dois vetores de tamanho 4. 

\begin{figure}[!htb]
\centering
\includegraphics[width=9cm]{img/merge2.jpg}
\label{fig:merge2}
\end{figure}

Isso não altera a seqüência dos itens. Agora dividimos esses dois vetores em metades. 

\begin{figure}[!htb]
\centering
\includegraphics[width=9cm]{img/merge3.jpg}
\label{fig:merge3}
\end{figure}

Nós dividimos ainda mais esses vetores e alcançamos o valor único que não pode mais ser dividido. 

\begin{figure}[!htb]
\centering
\includegraphics[width=9cm]{img/merge4.jpg}
\label{fig:merge4}
\end{figure}

Agora, nós os combinamos exatamente da mesma maneira como foram divididos. Observe as cores dadas a essas listas. Primeiro comparamos o elemento de cada lista e, em seguida, os combinamos em outra lista de maneira ordenada. Vemos que 14 e 33 estão em posições ordenadas. Comparamos 27 e 10 e na lista de destino de 2 valores colocamos 10 primeiro, seguido por 27. Alteramos a ordem de 19 e 35, enquanto 42 e 44 são colocados sequencialmente. 

\begin{figure}[!htb]
\centering
\includegraphics[width=9cm]{img/merge5.jpg}
\label{fig:merge5}
\end{figure}

Na próxima iteração da fase de mesclagem, dois índices iteram sobre os conjuntos, de forma que se compara o menor elemento corrente de um conjunto com o menor elemento corrente do outro. Dessa forma, ao retirar os elementos um a um, de forma ordenada, obtém-se uma nova sub-lista ordenada.

\begin{figure}[!htb]
\centering
\includegraphics[width=9cm]{img/merge6.jpg}
\label{fig:merge6}
\end{figure}

Após a última mesclagem, a lista original encontra-se ordenada

\begin{figure}[!htb]
\centering
\includegraphics[width=9cm]{img/merge7.jpg}
\label{fig:merge7}
\end{figure}

\section{Tim Sort}

O Tim Sort foi implementado pela primeira vez em 2002 por Tim Peters para uso em Python. Supostamente, veio do entendimento de que a maioria dos algoritmos de classificação nascem em salas de aula e não são projetados para uso prático em dados do mundo real. Tim Sort tira proveito de padrões comuns em dados e utiliza uma combinação de Merge Sort e Insertion Sort junto com alguma lógica interna para otimizar a manipulação de dados em grande escala.

A chave para entender a implementação de Tim Sort é entender o conceito de \textit{runs}. O Tim Sort aproveita os dados pré-classificados que ocorrem naturalmente a seu favor. Por pré-ordenado, simplesmente queremos dizer que os elementos sequenciais estão todos aumentando ou diminuindo (não nos importamos com quais). Em um passo a passo, o algoritmo pode ser descrito da seguinte maneira:

\begin{itemize}

\item Estabeleça um tamanho de minrun que seja uma potência de 2 (geralmente 32, nunca mais que 64 ou seu Insertion Sort perderá eficiência)
\item Encontrar uma run, uma sub-lista que se estenda enquanto os elementos estiverem ordenados, mas que tenha, pelo menos a quantidade minrun de elementos.
\item Se a run não tiver a quantidade mínima do minrun, usar o Insertion Sort para pegar os itens subsequentes ou anteriores e inseri-los na run até que tenha o tamanho mínimo correto.
\item Repita até que todo o vetor seja dividido em subseções classificadas.
\item Usar o merge do Merge Sort para unir os vetores ordenados. 


\end{itemize}

\section{Análise de Complexidade}

A análise de complexidade do algoritmo é projetada para comparar dois algoritmos em um nível teórico - ignorando detalhes de baixo nível, como a linguagem de programação, o hardware em que o algoritmo é executado ou o conjunto de instruções de uma determinada CPU. O objetivo é comparar algoritmos em termos das operações relevantes que são executadas na rotina. Um algoritmo ruim escrito em uma linguagem de programação de baixo nível, como assembly, pode executar muito mais rápido do que um bom algoritmo escrito em uma linguagem de programação de alto nível, como Python ou Ruby. Portanto, a análise de complexidade provê um parâmetro do que realmente é um "algoritmo melhor". 

Essa análise é feita em 3 cenários: melhor caso, pior caso e caso médio. A ocorrência de um cenário particular depende do formato da entrada a ser processada. No caso do Insertion Sort, por exemplo, o pior caso se dá quando a entrada está em ordem decrescente. Esse cenário faz com que o algoritmo percorra, no \textit{loop} interno, todas as posições possíveis para fazer a inserção do elemento. Como esse laço de repetição está dentro de uma iteração que percorre os N elementos do vetor, é dito que nesse pior caso a complexidade é O(n\textsuperscript{2}). A tabela abaixo exibe a complexidade dos algoritmos analisados e de antemão provê uma referência do seu comportamento à medida que o tamanho da entrada crescer durante a execução.

\begin{figure}[!htb]
\centering
\includegraphics[width=10cm]{img/complexidade.png}
\caption{Análise de complexidade dos algoritmos em estudo}
\label{fig:complexidade}
\end{figure}


% ----------------------------------------------------------
% PARTE - preparação da pesquisa
% ----------------------------------------------------------
%\part{Metodologia} %Preparação da Pesquisa

% Aqui entram os capítulos relacionados à metodologia da pesquisa a ser realizada
\chapter{Predição de sucesso utilizando a nota de avaliação do IMDB}

A hipótese a ser analisada no processo pode ser descrita como: "para cada nó, é possível predizer sua nota com base nas notas dos vizinhos (ou seja, aqueles que atuaram com ele em algum momento)?". Em um teste inicial de viabilidade, verifica-se uma alta correlação entre a nota do ator e a média dos vizinhos utilizando a correlação de Pearson. O cálculo apontou uma correlação de 0,96 entre o conjunto de valores de notas dos vértices e a média das notas dos seus vizinhos imediatos. Essa alta correlação é consequência também da formação dos cliques, já que o vértice possui muitos vizinhos imediatos com uma nota próxima (ainda ponderado pela quantidade de elementos da vizinhança).

Utilizando a média como forma direta de predição, e comparando com a nota efetiva, é possível obter uma estimativa da acurácia. Neste caso, como forma de avaliação, é calculado o erro quadrático entre o valor predito e o real, e para toda a rede, obtém-se o erro médio. Neste caso, o erro médio obtido foi de 0,25 em notas que variam de 0 a 10. Ainda que o erro seja baixo, uma opção de modelo para predição é utilizar os conjuntos de valores para definir uma função linear que possa predizer melhor a nota com base na média. 

\begin{figure}[!htb]
\centering
\includegraphics[width=15cm]{img/funcaolinear.png}
\caption{Função linear para melhor aproximação do conjunto de dados}
\label{fig:funcaolinear}
\end{figure}

A figura \ref{fig:funcaolinear} ilustra a dispersão dos valores e a reta que melhor estima os valores. Utilizando regressão linear, a função linear de variável simples que descreve essa reta é dada por f(x) = 1.22x - 1.47. Utilizando essa função para fazer a predição, o erro médio cai para 0,21(...). 
Indo um pouco além, outra hipótese é considerar um modelo multi-variável que considere as médias de vizinhos a distâncias N do vértice. Por exemplo, para N=2, a média dos vizinhos dos vizinhos também são consideradas para tentar ajustar melhor o modelo. Efetuando o primeiro teste para modelo multi-variável com N=2, obteve-se a função f(x) = 1.29x\textsubscript{1} + - 0.26x\textsubscript{2} - 0.19 (onde x\textsubscript{1} é a média dos vizinhos diretos, enquanto x\textsubscript{2} é a média dos vizinhos dos vizinhos). No entando, ao efetuar o teste de predição, o erro médio obtido foi da mesma ordem de 0.21(...), não justificando seu custo computacional.

Utilizando então a função linear obtida para predição, é possível tomar o seguinte cenário hipotético para ilustrar: um ator X, não conhecido previamente, mas que se sabe que atuou com Will Smith, Adam Sandler, Mila Kunis, Keanu Reeves, Rodrigo Santoro e Carrie Fisher (ilustrado na figura \ref{fig:predict}. O valor predito da sua média é obtido substituindo a variável na função pela média desses vizinhos. Esse valor seria dado então por 1,22 * 6,31 - 1,47 = 6,23.


\begin{figure}[!htb]
\centering
\includegraphics[width=16cm]{img/predict.png}
\caption{Cenário utilizando na predição}
\label{fig:predict}
\end{figure}



% ----------------------------------------------------------
% Resultados
% ----------------------------------------------------------
%\part{Resultados}

%Aqui entram os capitulos contendo a análise e discussão dos resultados. A conclusão fica logo abaixo, em um capítulo sem numeração
% ---	
% primeiro capitulo de Resultados
% ---
%\chapter{Discussão da análise e dos resultados}


% ---
% Finaliza a parte no bookmark do PDF, para que se inicie o bookmark na raiz
% ---
\bookmarksetup{startatroot}% 
% ---

% ---
% Conclusão
% ---
\chapter*[Conclusão]{Conclusão}
\addcontentsline{toc}{chapter}{Conclusão}

O Tim Sort, como algoritmo híbrido, tira vantagem do fato de que, em aplicações no mundo real, frequentemente são encontrados padrões nos quais sub-conjuntos de uma lista já estão ordenados. Se beneficiando dessa característica e fazendo uso das funções de ordenação do Insertion e Merge Sort para uma estratégia de divisão e conquista, o algoritmo se mostra muito eficiente aos cenários em que é submetido na prática e acabou se tornando o método padrão das linguagens Python e Java SE 7.

Para trabalhos futuros, sugerem-se algumas otimizações que visam melhorar ainda mais o desempenho do Tim Sort, como por exemplo a ordenação por inserção com busca binária e o merge com galopeamento binário. A implementação aqui demonstrada na linguagem python, ainda que menos verbosa e de mais fácil compreensão, é menos eficiente do que uma linguagem de mais baixo nível, como C. Essa transcrição também é sugerida como melhoria em termos de performance.

Por fim, o trabalho aqui proposto é também uma demonstração de análise e comparação de diferentes algoritmos para solução de um mesmo problema. A partir da análise teórica de complexidade das rotinas propostas, a verificação prática se dá para confirmar o comportamento assintótico dessas soluções em diferentes contextos à medida que são sobrecarregados com entradas de diversas escalas. Em um caso de uso real, esse processo vem apoiar a tomada de decisão de acordo com o cenário em questão.

% ----------------------------------------------------------
% ELEMENTOS PÓS-TEXTUAIS
% ----------------------------------------------------------
\postextual

% ----------------------------------------------------------
% Referências bibliográficas
% ----------------------------------------------------------

%\bibliography{referencias}

% ----------------------------------------------------------
% Glossário
% ----------------------------------------------------------
%
% Consulte o manual da classe abntex2 para orientações sobre o glossário.
%
%\glossary

% ----------------------------------------------------------
% Apêndices - Caso não existam, comentar
% ----------------------------------------------------------

% ----------------------------------------------------------
% Anexos - Caso não existam, comentar
% ----------------------------------------------------------
%\include{08_anexos}

%---------------------------------------------------------------------
% INDICE REMISSIVO
%---------------------------------------------------------------------
\printindex


\begin{thebibliography}{9}



\bibitem{b2}\textbf{Insertion Sort Algorithm}. Disponível em: <\href{https://www.tutorialspoint.com/data\_structures\_algorithms/insertion\_sort\_algorithm.htm}{https://www.tutorialspoint.com/data\_structures\_algorithms/insertion\_sort\_algorithm.htm}>. Acesso em março de 2021.

\bibitem{b2}\textbf{Merge Sort Algorithm}. Disponível em: <\href{https://www.tutorialspoint.com/data\_structures\_algorithms/merge\_sort\_algorithm.htm
}{https://www.tutorialspoint.com/data\_structures\_algorithms/merge\_sort\_algorithm.htm
}>. Acesso em março de 2021.

\bibitem{b2}\textbf{Tim Sort Algorithm}. Disponível em: <\href{https://www.geeksforgeeks.org/timsort/}{https://www.geeksforgeeks.org/timsort/}>. Acesso em março de 2021.

\bibitem{b2}\textbf{An Introduction to Algorithm Complexity Analysis}. Disponível em: <\href{https://discrete.gr/complexity/}{https://discrete.gr/complexity/}>. Acesso em março de 2021.




\end{thebibliography}

\end{document}

