\chapter*[Conclusão]{Conclusão}
\addcontentsline{toc}{chapter}{Conclusão}

O Tim Sort, como algoritmo híbrido, tira vantagem do fato de que, em aplicações no mundo real, frequentemente são encontrados padrões nos quais sub-conjuntos de uma lista já estão ordenados. Se beneficiando dessa característica e fazendo uso das funções de ordenação do Insertion e Merge Sort para uma estratégia de divisão e conquista, o algoritmo se mostra muito eficiente aos cenários em que é submetido na prática e acabou se tornando o método padrão das linguagens Python e Java SE 7.

Para trabalhos futuros, sugerem-se algumas otimizações que visam melhorar ainda mais o desempenho do Tim Sort, como por exemplo a ordenação por inserção com busca binária e o merge com galopeamento binário. A implementação aqui demonstrada na linguagem python, ainda que menos verbosa e de mais fácil compreensão, é menos eficiente do que uma linguagem de mais baixo nível, como C. Essa transcrição também é sugerida como melhoria em termos de performance.

Por fim, o trabalho aqui proposto é também uma demonstração de análise e comparação de diferentes algoritmos para solução de um mesmo problema. A partir da análise teórica de complexidade das rotinas propostas, a verificação prática se dá para confirmar o comportamento assintótico dessas soluções em diferentes contextos à medida que são sobrecarregados com entradas de diversas escalas. Em um caso de uso real, esse processo vem apoiar a tomada de decisão de acordo com o cenário em questão.