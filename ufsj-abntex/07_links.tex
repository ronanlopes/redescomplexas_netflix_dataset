\chapter*[Conclusão]{Conclusão}
\addcontentsline{toc}{chapter}{Conclusão}

O estudo aqui desenvolvido, bem como os resultados obtidos, mostraram que é viável a utilização de Sistemas Gerenciadores de Conteúdo para promoção de acessibilidade em \textit{websites} de característica dinâmica. Como tem crescido o número de plataformas que onde faz-se necessário o auxílio na gestão do conteúdo, a validação do uso de um SGC para conteúdo acessível vem apoiar o desenvolvimento de aplicações onde a acessibilidade é requisito. 

Além dos \textit{websites} de administração pública, a promoção de acessibilidade web é desejável mesmo em contextos onde esta não é condição obrigatória. Além do cumprimento de responsabilidade social, tornar acessível uma página na internet amplia o público leitor do conteúdo publicado. Ademais, com a característica dinâmica e as novas plataformas que surgem na web, estes usuários não são apenas leitores passivos, mas também interagem com esses sistemas e geram conteúdo a partir dessas interações.

Para trabalhos futuros, sugere-se a correção dos erros de acessibilidade remanescentes no portal. Além disso, a promoção de acessibilidade web deve estender-se também aos sistemas internos da universidade, que também são constantemente utilizados por alunos e funcionários. Durante o trabalho desenvolvido identificou-se também, através do estudo para implementação, que o código-fonte do portal atual se apresenta mal estruturado, uma vez que está pouco modularizado. Faz-se necessário então um \textit{refactoring} nesse código, a fim de auxiliar futuras manutenções e alterações.