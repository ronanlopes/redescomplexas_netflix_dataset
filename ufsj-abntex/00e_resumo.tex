% resumo em português
\begin{resumo}

No último dia do ano de 2019, a China fez um comunicado à Organização Mundial de Saúde (OMS), alertando para o surgimento de uma série de casos de pneumonia de origem desconhecida, que vieram a ser reconhecidos como casos do corona vírus SARS-CoV-2, que provocam a infecção nomeada COVID-19. Desde então, com a alta capacidade de dispersão do vírus que se espalhou pelo globo, o assunto tomou o primeiro lugar nos tópicos mais importantes devido à sua relevância e seu impacto no âmbito político-social. Sendo as redes sociais um espelho virtual dos eventos do cotidiano, naturalmente cresceu também em ritmo exponencial o volume de informação propagado sobre o assunto. Esse grande volume descentralizado de dados pode apresentar em seu formato um empecilho em integrar e dar alguma interpretação a essas informações. Nesse cenário, a ciência de dados surge como uma área de estudo multidisciplinar que, através de métodos de mineração, análise estatística, aprendizado de máquina e visualização de dados, vem apoiar a tomada de decisão nesses contextos. O estudo aqui desenvolvido se propõe coletar e integrar diferentes fontes de dados e, a partir dessas informações, gerar funcionalidades visuais que dão origem à aplicação web COVIEWER-19.

 \vspace{\onelineskip}
    
 \noindent
 \textbf{Palavras-chaves}: corona vírus, covid-19, ciência de dados, aprendizado de máquina, mineração de dados
\end{resumo}

% resumo em inglês
\begin{resumo}[Abstract]
 \begin{otherlanguage*}{english}
\textit{On December 31th, 2019, China country made a statement to the World Health Organization (WHO), warning about a series of pneumonia cases of unknown source, that came to be recognized as the SARS-CoV-2 corona virus, which causes the infection named COVID-19. Since then, as the virus has a high dispersion (spreading across the globe), this subject became the most important topic in the world due its influence on social political. Once social networks are a virtual mirror of day-to-day events, that subject related data also grew in a exponential pace on those platforms. That large and decentralized volume of data can be a challenge to merge and give a meaning on an analytical point of view. In scenarios like this, data science supports decision making through methods of data mining, statistical analysis, machine learning and data visualization. The study developed here is about collecting and merging different data sources and, for those databases, generate visual resources that will become the features of COVIEWER-19 web application. }

   \vspace{\onelineskip}
 
   \noindent 
   \textit{\textbf{Key-words}: corona virus, covid-19, data science, machine learning, data mining }
 \end{otherlanguage*}
\end{resumo}